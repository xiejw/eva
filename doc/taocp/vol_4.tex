\input doc/format
\input eplain
\notespagesize
%
Volume 4
%
\section{Horn Clause and Formula}

A Horn clause is a clause (a disjunction of literals) with at most one positive
literal.  A Horn formula is a propositional formula formed by conjunction of
Horn clauses.

\subsection{Definite Horn Formula and The Core}
The definite Horn formula
must satisfy
$$
f(1, 1, ..., 1) = 1.
$$
The core of the definite Horn formula is set the variables which must be true
whenever $f$ is true.

\subsection{Algorithrm C}

The algorithm is as follows
\numberedlist
\li Put all positive literal in the single variable clause into core.  They must
    be true.
\li Keep deducing the proposition of non-positive literal, once their values
    are known to be true, i.e., in core. Once all non-positive literals in a
    clause are deduced, its positive literal, if present, must be in core.
\endnumberedlist

\subsection{Indefinite Horn Formula}

Exercise 48 provides the steps to test satisfiability of Horn formula in
general. The idea is quite simple

\numberedlist
\li Introduce a new variable $\lambda$, and convert all indefinite
causes to definite cause. For example, the following indefinite Horn clause
$$
\bar a \vee \bar b  % \wedge
$$
will be converted as
$$
\bar a \vee \bar b \vee \lambda.
$$
\li Apply Algorithrm C to the new definite Horn formula. The original
Horn formula is satisfiable if and only if `lambda` is not in the core of the
new definite Horn formula.
\endnumberedlist

\bye
